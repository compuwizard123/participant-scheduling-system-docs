\documentclass{article}
\usepackage{graphicx}
\begin{document}
\begin{titlepage}
\begin{center}
\includegraphics[width=0.15\textwidth]{images/rh}\\[1.0cm]
\textsc{\large Rose-Hulman Institute of Technology}\\[1.5cm]
\includegraphics[width=0.75\textwidth]{images/pss}\\[1.0cm]
\textsc{\large Team Engineering Journal}\\[1.0cm]
\large Trey Cahill \hspace{0.2cm} Katie Greenwald \hspace{0.2cm} Samad Jawaid \hspace{0.2cm} Kevin Risden
\end{center}
\end{titlepage}
\newpage

\section{Thursday, December 8, 2011} % Katie
In the week of November 28th, we got back into the swing of working on our project. I
personally had not worked on this project before, so one of the main goals for that
week was to get me up to speed. As we completed the compilation document of Milestone
1, I reviewed what the project was and how we were expecting to implement it. We did
not have our weekly meeting with our client or project manager, but we did set times
for those meetings. After our regular meeting with Dr. Chenoweth, we had several
action items to take care of. Specifically, we needed to discuss the scope of our
project with our client to make sure we were all on the same page. We also needed to
get me ready to go with Latex, Redmine, and Git.

\section{Friday, December 9, 2011} % Katie
In this past week, we continued bringing me up to speed, and began our work on
Milestone 2. On Monday, we met with our client, Allie, and I was introduced. We also
made sure that all of us had the same understanding with regards to the scope of the
project. We were also given a feature request; Allie and her co-workers wanted to have
the ability multiple people sign up for any given time slot, up to a maximum set at
the event's creation. We replied that we would give her our answer at the next
meeting. Later in the week, we decided that the feature would be trivial to implement,
so we will be accepting the request. In our first project manager meeting, we discussed
how the quarter was going to go. We decided that, in general, we would turn our
milestones in to our PM a week before they are due, so that we have time to edit them
before turning them in. We also worked on Milestone 2 and installed Git and Latex on
my computer. For next week, we are to finish the milestone and get Redmine working for
me.

\section{Thursday, December 8, 2011} % Samad
Today, we wrote our initial deployment guide for our client, Allie. The
introduction is below.
\begin{quote}
This document will provide a very good sense of how to deploy the Participant
Scheduling System, but it will not be a definite guide. Though most instructions
include concrete examples, some steps may be missing. Pay special attention
when configuring Nginx, Apache, and PostgreSQL. Furthermore, some values will
need to be changed, like the references to \verb#rose-hulman.edu#. Good luck!
\end{quote}

\section{Friday, December 16, 2011} % Samad
Last night, we received a response from our client, Allie, regarding our
initial deployment guide. We expected her to have a server similar to our
CSSE virtual machine for our prototype. It turns out that she does not
have a full virtual machine; rather, she just has an AFS folder through
CSL (her IAIT). We have not yet decided how to proceed, but hopefully,
this will not be a huge hurdle.

\section{Friday, January 6, 2012} %Trey
Today, we decided upon who will work on what for milestone 3. Samad worked on the Package Diagram
and revising and trouble shooting the deployment guide.  I Worked on  the Sequence diagram, Operation
contracts and interaction diagram for browsing an experiment and signing up for an experiment.  Katie
did the same work as I did, but for creating and modifying an experiment.  Kevin also did the same, but for 
creating an account and login.  As a team we produced a Design Class Diagram and will continue working
on parts of this milestone to make it better.

\section{Monday, January 9, 2012} %Trey
Today, we meet with Allie to work on setting a meeting time and make contact in the new year.  We set a
meeting time for our normal time (Mondays at 10:50 am Est) until told otherwise.  We talked about
possible changes for deployment and also a possibility of getting login information to work on the deployment
ourselves if need be.  We asked if there were any changes that she could think of, but she had nothing for us.
We ended the meeting by finishing putting together our rough draft of Milestone 3 for submition to our Project
Manager.  We also discussed who will be taking what tickets in Redmine, but not much came of this since it was 
not on our agenda.

\section{Monday, January 23, 2012} %Kevin
As usual we met with Allie for our weekly client meeting. We have been working with her to setup a vm for the
deployment of our web application. This process has been taking a lot longer than we anticipated. Part of this 
is due to the fact that we misunderstood what she meant on week 1 of the fall quarter. She stated that she had
a machine to run the system on and gave us a list of version numbers. Based on this and her wording, we 
determined that it was a full virtual machine on which she had rights. Instead, she was intending to run the 
server on her local account on a department computer. Since we discovered this, we had to convince her to get a
full virtual machine from CSL (their IAIT) and this process has taken over 2 weeks so far. She is going to 
contact them again this week to determine the status. We have a deployment guide that we know works so once we 
are able to get Allie setup with a virtual machine we will be able to move quickly into the deployment phase.

\section{Thursday, January 26, 2012} %Kevin
This week has been focused primarily on implementing secondary features and making some minor bug fixes that
we had found earlier in the project. One of the secondary features that we were focusing on was how to make 
the experiments table look better. The reason behind this was that the table has a lot of information that 
can be sorted by and this was a requested feature by Allie. We determined that there were a few options 
regarding the layout of the table and the method of which to sort. Samad intended to try to do the work on the
back end in python and return it to the browser. I suggested that we use an open source jQuery JavaScript 
library that I had worked on and with at my internship. The library enables you to make any HTML table into a 
sortable, searchable table with ease. This required a bit more knowledge to setup but was not beyond the 
ability of our group. We were able to implement this feature and not spend as much time have to come up with
a solution to do it ourselves.

\section{Thursday, February 9, 2012} % Samad
Our client finally reported back to us with details about her
production server. Over nineteen weeks ago, she led us to believe that
our web application would be deployed on a Fedora virtual private
server. None of us were too familiar with Fedora, but we prototyped on
a Fedora virtual private server anyway. Even though we preferred
Ubuntu, we decided it was more important to match the client's
specifications than our preferences. Everything works very well on our
prototype, and we even wrote a step-by-step deployment guide for this
setup.

With one week remaining in the project, some major obstacles have been
placed before us; for example, the client actually does not have a
virtual private server! Hence, we cannot use all of the tools we
initially planned to use like nginx for static content or mod\_wsgi.
Furthermore, our client does not even have sudo privileges on her
``server.'' We are hoping that we can make the required adjustments to
our deployment plan in the remaining time. It would be a travesty for
our project to not be used, not because it was not awesome software
but because our client could not deploy it.

\section{Friday, February 10, 2012} % Samad
For the sake of focusing on deploying our project, we have had to make
a difficult decision to not finish a few minor coding tasks. First,
there are two small bugs that will not be addressed: one is that dates
and times are sorted alphabetically and therefore incorrectly in our
tables, and the other is that {\tt ExperimentDateTimeRange}s are not
completely validated in the administration, part of a very rare use
case. Additionally, when researchers modify or delete {\tt
Experiment}s, {\tt ExperimentDate}s, or {\tt
ExperimentDateTimeRange}s, they are not warned of deleting related
{\tt Slot}s or {\tt Appointment}s. Researchers also wanted the ability
to create {\tt Building}s\textbackslash{\tt Room}s and {\tt
Qualification}s themselves, but instead an administrator will have to
do so for them. Finally, allowing researchers to mass-delete {\tt
Experiment}s and view all {\tt Appointment}s in a calendar will not be
implemented. We sincerely believe that none of this will detract from
the overall usefulness of our product.


\end{document}
