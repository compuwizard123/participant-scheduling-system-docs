\section{Product Overview}
This section provides a high-level view of the product capabilities, interfaces to other applications, and system configurations.

\subsection{Product perspective}
The participant scheduling system will be a new product. It will be used to schedule experiments and participants in the Human-Computer Interaction Lab at the University of Wisconsin-Madison. The product is independent and totally self-contained, besides a few external software packages; it is not a component of a larger system.

\subsection{Elevator Statement}
For the researchers in the Human-Computer Interaction Lab at the University of Wisconsin-Madison who currently schedule experiments and participants with rudimentary tools such as pencil and paper, email, or Google Calendar, the participant scheduling system will be a web application that will streamline the lab's scheduling process. Unlike current solutions, this application will be the same for every researcher, so it will also be easier for participants to be a part of multiple experiments.

\subsection{Summary of Capabilities}
Here are the major benefits and features the product will provide.
\begin{table}[!h]
    \begin{tabular}{|l|l|}
        \hline
        Customer Benefit & Supporting Feature \\ \hline
        List of participants for an experiment & Reports \\ \hline
        Room availability (avoid conflicts) & Overall lab schedule \\ \hline
        Simple sign up & Intuitive user interface \\ \hline
        Track all experiments & Experiments manager \\ \hline
        Access from anywhere at any time & Web application \\ \hline
    \end{tabular}
\end{table}
\subsection{Assumptions and Dependencies}
\begin{itemize}
\item The participant scheduling system will be a web application.
\item The server has the necessary operating system and software.
\item There is no integration with any other system.
\item There is no import of existing data.
\end{itemize}
\subsection{Rough Estimate of the Cost}
There is no monetary cost for this project, because the software development, as part of a college class, is free. Similarly, all software used is open-source. Furthermore, the client will be provided with free servers through the University of Wisconsin-Madison for the finished product. The client will perform maintenance and management on their own.
