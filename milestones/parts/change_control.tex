\section{Change Control}
\subsection{Change Requests}
The team will receive requests by either email or verbal request from the client.  Information required from the client would be a description of the change and the version in which the change should be implemented.  Should a change be made internally, the change request will be received during a meeting time. A ticket will then be created in \createindex{Redmine} detailing each change requested.

\subsection{Managing change}
The team will manage change via \createindex{Redmine}. Each change will be accepted or denied after meeting with the team. From this point the team will determine what is affected and add the needed information to \createindex{Redmine}. Also any information that would change in the documentation will also be altered. If the client requests a change, the team will discuss specifics of the change such as effort required and who/where the change will be implemented to determine if the change will be accepted or denied. If a team member requests a change then the request will be reviewed by the team during the next team meeting.  If the request is accepted, the change will follow the flow described above; otherwise, the change will be considered dead.

\subsection{Changes to Project Artifacts}
The team currently uses \createindex{GitHub} for both documentation and source code.  Since documentation and source code are changed frequently, team members create a new branch to do their work in when making changes.  After finishing their work, a team member would merge their branch to the master branch.
