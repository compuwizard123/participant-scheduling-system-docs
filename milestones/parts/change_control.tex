\section{Change Control}
\subsection{Change Requests}
The team will receive requests by either email or verbal request from the client.  Information required from the client would be a description of the change and the version in which the change should be implemented.  Should a change be made internally, the change request will be received during a meeting time.  All information that would be required for Redmine will be needed.

\subsection{Managing change}
The team will manage change via Redmine.  Each change will be accepted or denied after meeting while meeting with the team.  From this point the team will determine what is effected and add the needed information to Redmine.  Also any information that would change in the documentation will also be altered.  If the client requests a change and the change is within the allowed scope of change, the team will automatically accept the change and discuss specifics of the change such as effort required, who and where the change will be implemented, etc. If a team member requests a change then the request will be reviewed by the team during the next team meeting.  If the request is accepted, the change will follow the flow described above; other wise the change will be considered dead.

\subsection{Changes to Project Artifacts}
The team currently uses GitHub for both documentation and source code.  Since documentation and source code are changed frequently, team members create a new branch to do their work in when making changes.  After finishing their work, a team member would merge their branch to the master branch.  By merging a branch, instead of working on the master branch, the team is protected from two people working on the same document and then a team member losing the changes they made.
