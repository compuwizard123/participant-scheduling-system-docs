\section{GRASP Principles}
\subsection{Creator}

\subsection{Info Expert}

\subsection{Controller}

\subsection{High Cohesion}
High cohesion is achived in many parts of our code, but an example of where cohesion is extremely strong is in the Administration code.  All of our Administration  procedures, fields, responsiblities, and all similar objects are located in the administration package, specifically in the django.contrib.admin package as seen in the package diagram in section 4 of this document.  By having administration as its own package, it leaves other administration tasks out of other unrelated classes in our system.

\subsection{Low Coupling}
By seperating out objects and responiblities into our their own packages, we achive a fairly low coupling.  Each seperate package contains a resonibility and is conntected only to the classes to which must be connected.  Our class diagram in section 5, shows how our classes interact with each other and also shows that each class is used only by a class that needs its information.  For example, Qualification is only connected to Experiment; if could possibly be connected to participant, but because you will not be able to completely verify the particiapants qualifications until they show up for the experiment, there is no need to make the connection between Participant and Qualifications.

\subsection{Pure Fabrication}
Pure Fabrication becomes uses in our system by creating ExperimentDateTimeRange and ExperimentDate.  Neither of these two classes are in the domain model, but since they make the code easier to work with and seperate out responsiblity, increasing cohesion, they become very useful as classes.  Should we not have these two classes Slot or Experiment would have to contain this information, which would decrease cohesion and generally add to the complexity of the Slot or Experiment or both.

\subsection{Indirection}
The GRASP principle indirection directly relates to our system for how we need to represent experiment dates and the experiment datetime ranges. An experiment must keep track of the dates and time slots for each day that it is offered. We decided that having two intermediate classes, ExperimentDate and ExperimentDateTimeRange, would reduce coupling and ensure easier maintainability of the system. The ExperimentDate class keeps track of the slots that the ExperimentDateTimeRange can generate. This enables the user to enter a time range and the system will then calculate the specific time slots. The Experiment class just has to have the ExperimentDates. This makes the coupling of time slots to experiments cleaner. An alternative would be for the experiment to have a massive list of all the dates and slots that it is offered. This would make it difficult to add or remove slots later and not know which slots are current filled participants.

\subsection{Polymorphism}
Currently, the Participant Scheduling System requires both researchers and participants. In order to accomplish this, a User class was introduced to provide a standard base class and then the Researcher class was derived from this. This provides our solution with the polymorphism GRASP principle. The other option was to create two separate classes for researcher and user, but then there would be duplicated code. Furthermore, if there needs to be another type of user then it will be simpler to just extend the current User class.

\subsection{Protected Variation}
In our system, protected variation builds off our decision for polymorphism. The User class enables the protected variation GRASP principle since it protects us from changes in the type of users who need to use the system. If the client comes back with a request for another type of user besides participant and researcher, the system is setup to handle this by just extending the User class. This provides the most elegant solution to the problem since the other option would have been to create the different user classes separately and would make it difficult to extend later.