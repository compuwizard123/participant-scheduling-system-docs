\section{Introduction}
The Human-Computer Interaction Lab at the University of Wisconsin-Madison wants a web-based system to better manage the scheduling of participants for their studies.  These studies range from one-on-one experiments to group interactions, and many of them involve the robot used by the lab.  Currently, each researcher arranges studies independently via email and is responsible for scheduling rooms, avoiding conflicts, and notifying participants of changes; unifying this information onto one system simplifies all of these tasks.  To the client, the most important benefit of a unified system is the ability for participants to easily browse all available experiments, which is not possible over email.  However, a variety of other functionality should be integrated into this utility to take advantage of the unity of information; most notable is recognizing room conflicts when scheduling studies, since the lab has only one robot and it cannot be moved.\cite{website:HCI}

Project information will be documented as follows:  Milestone 1 provides an overview of the project, from client background to key features and requirements.  Milestone 2 covers the behavior of the system, including use cases and data flow diagrams.  Milestone 3 details constraints, back-end requirements, and elaborates upon the user interface.  Testing and maintenance information can be found in Milestone 4.  Milestone 5 will include usability data and interface re-design related to such data.