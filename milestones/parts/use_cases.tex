\section{Use Cases}

\subsection{Authentication}
\begin{outline}[enumerate]

\1 {\bf Name: Authentication}
\2 Brief Description: User logs in, logs out, or creates an account.
\2 Actors: User
\2 Basic Flow:
\3 Step 1: User clicks the "login/logout/create account" button from any page.
\4 The system will record where the user entered this use case from, as well as any entered information.  This is so the system might seamlessly return the user to what they were doing after authentication is complete.
\4 Alternate Flow: If the user is already logged in, the system will log them out.  They are no longer considered authenticated and may not be able to view their previous page; if so, they will be shunted to the main page.  Exit the use case here.
\3 Step 2: User prompted for Email and Password.  There is a button labeled "Create New Account" which will alter the interface, displaying instructions and revealing the fields Name, Phone, and Confirm Password.  After completing these fields, user clicks "Submit" button.
\4 If the user didn't expand the "Create New Account" option, the system sends their login information to the database.  If it matches, continue to step 3.
\4 Alternate Flow: If the database didn't recognize the email, return to step 2 and expand "Create New Account".
\4 Alternate Flow: If the password didn't match the one stored in the database, return to step 2 and display a message informing the user their password was incorrect.
\3 If the "Create New Account" area was open, the system will create a new account.
\4 If the pasword and confirm password fields don't match, return to step 2 and display a message saying so.  Keep all entered information except the password and confirm password fields.
\4 Query the database to see if it already has an account with that email.  If it does, return to step 2 display a message saying so.
\4 Send an email to the address listed in the Email field.  If this fails, return to step 2 and display a message saying so.
\4 Send all entered information to the database to be added.  Continue to step 3.
\3 Step 3: System displays a message confirming account creation or successful login.  The user is now logged in.  The user will be returned to the page they entered authentication from automatically after 10 seconds, or can click a link to do so immediately.
\2 Pre-conditions:
\3 System is functional.
\2 Post-conditions:
\3 User is logged in or out, the opposite of what they were before.
\2 Special Requirements:
\3 N/A
\2 Feature Mapping:
\3 Levels of Authentication
\3 Accounts

\end{outline}

\subsection{Appointments}
\begin{outline}[enumerate]


\1 {\bf Name: Select Experiment}
\2 Brief Description: Participant views and selects experiment to join
\2 Actors: Participant (henceforth "user")
\2 Basic Flow: (User can return to homepage via their browser at any time.)  Each page has a login/logout button, which will follow the use case "Authentication" if clicked.
\3 Homepage: The system homepage has a table displaying currently running experiments.
\3 Step 1: User can sort or filter experiment table.
\3 Step 2: User clicks an experiment.  The system navigates them to that experiment's page.
\3 Experiment page: Each experiment has a webpage with its name, description, and a list of timeslots, as well as a button to join the experiment.
\3 Step 3: User reads experiment description and required qualifications.
\3 Step 4: User can sort or filter timeslot list.
\3 Step 5: User clicks "join experiment" or a timeslot button.  This takes them to use case "sign up for experiment".
\2 Alternate Flows:
\3 User decides to view a different experiment by navigating with their browser or clicking a button on any page.  They are returned to the homepage.
\2 Pre-conditions:
\3 System is functional.
\3 There is at least one experiment currently offered.
\2 Post-conditions:
\3 User has clicked "join experiment" or a timeslot button for some experiment.

\1 {\bf Name: Sign up for Experiment}
\2 Brief Description: Participant enters data and confirms appointment.
\2 Actors: Participant (henceforth "user")
\2 Basic Flow:
\3 Step 1: If the user is not currently logged in, run use case "Authentication".  They must be logged in to continue.
\3 Confirmation Page: This page is available only while logged in.  It displays experiment name, required qualifications, and a checkbox for the user to verify they meet those qualifications. There is a list of timeslots. There is a "Confirm Appointment" button.
\3 Step 2: The user selects a timeslot (or lets the system do it for them if they did so in "Select Experiment") and checks the checkbox.
\3 Step 3: The user clicks the "Confirm Appointment" button.
\3 If there are no problems with the entered data, the system returns the user to the homepage and displays a message informing them of their successful registration.  The system will also send an email containing the experiment and timeslot information to the email account used to register.
\2 Alternate Flows:
\3 User attempts to "Confirm Appointment" before entering all required information/checking the checkbox.
\4 The system will return them to the confirmation page and inform them of what still needs to be done.
\3 User logs out while on the confirmation page.  The system will return them to the experiment page.
\2 Pre-conditions:
\3 System is functional.
\3 User has selected an experiment via the "Select Experiment" use case.
\3 The selected experiment has at least one viable timeslot.
\2 Post-conditions:
\3 Database has added appointment to user and experiment data.
\3 (Side effect) user is authenticated.
\2 Special Requirements:
\3 N/A

\1 {\bf Name: Sign Up for Experiment}
\2 Brief Description: Prospective participant chooses and signs up for an experiment.
\2 Actors: Participant (henceforth "user")
\2 Basic Flow: (User can return to a previous step with the "back" options on their browser.)  Each page has a login/logout button, which will follow the use case "Authentication" if clicked.
\3 Homepage: The system homepage has a button for login, and a table displaying currently running experiments.  Each experiment on the table can be clicked for further details.
\3 Step 1: User clicks on an experiment button.  This will navigate them to that experiment's page.
\4 Possible substep: The experiment listing table has options for sorting and filtering.  The user may click these buttons and enter filters without altering flow.
\3 Experiment Page: Each experiment has a webpage with its name, description, and a list of timeslots.  Each timeslot can be clicked to move to the confimation screen.  There is also a button to join the experiment which will send the user to the confirmation screen without a timeslot selected.
\3 Step 2: User clicks on a timeslot or the join button.  The system will navigate them to the confirmation page.
\4 Possible substep: The timeslot listing has options for sorting and filtering.  The user may click these buttons and enter filters without altering flow.
\3 Step 3: If the system attempts to move to the confirmation page but the user is not authenticated, it will follow the use case "Authentication" first.  Once they have successfully logged in, they will be sent directly to the confirmation page.
\3 Confirmation Page: This page can only be accessed while logged in.  It displays the experiment name, required qualifications, and a checkbox certifying that the user meets these requirements.  The timeslot list from the experiment page is also on this page, with the timeslot remaining selected if the user did so in step 2.  There is a large "Confirm Appointment" button at the bottom of the page.
\3 Step 4: The user checks the box, selects a timeslot if they have not already, and clicks the "Confirm Appointment" button.
\3 Final actions: The system will send the appointment data to the database and return the user to the homepage, with a popup confirming their successful registration.  The system will also send an email with the experiment and timeslot information to the account they used to register.
\2 Alternate Flows:
\3 User logs out while on the confirmation page.
\4 As a user cannot return to this page while logged out (the normal behavior for the Authentication use case), they will be shunted back to the experiment page.  Information provided in step 2 is not guaranteed to remain selected.
\3 Database reports an error when receiving appointment data.
\4 Unlikely to occur outside of concurrent modification (two people signing up at the same time), the system will shunt the user back to the confirmation page with the updated available timeslots and provide a popup explaining why this happened.
\2 Pre-conditions:
\3 System is functional.
\3 There is at least one experiment currently offered.
\2 Post-conditions:
\3 Database has added appointment to user and experiment data.
\3 (Side effect) User is authenticated.
\2 Special Requirements:
\3 N/A

\end{outline}
\subsection{Experiment Management}
\begin{outline}[enumerate]



% What about breaks in between slots?
\1 {\bf Name: Add Experiment}
\2 Brief Description: Experiments can be created by Administrators and Researchers
\2 Actors: Administrators and Researchers
\2 Basic Flow: (user can cancel at any time and follow Alt Flow 1)
\3 User must click on Add New Experiment link from the Administration "home" page
\3 System will display a screen with text boxes to enter experiment name, description,  and qualifications, multiple date/time choosers for the schedule times, and a drop down list to specify the length of the schedule slots
\3 User must enter the experiment information for name, description, qualifications, and schedule slots
\3 User can then begin setting up the schedule times by choosing date, begin, and end time for each slot they want to run the experiment
\3 User then must save the experiment by clicking the Save button
\3 System will then save the experiment to persistent storage and provide the user with confirmation that the experiment was created successfully and redirect user to all experiment view
\2 Alternate Flows:
\3 User cancels out of creating an experiment
\3 Saving an experiment fails
\2 Pre-conditions:
\3 User is an Administrator and/or a Researcher and has authenticated
\2 Post-conditions:
\3 System will have recorded the experiment or the system will notify the user why the creation of the experiment failed
\2 Special Requirements:
\3 N/A

\1 {\bf Name: Modify Experiment (user can cancel at any time and follow Alt Flow 1)}
\2 Brief Description: Experiments can be modified by Administrators and Researchers to change all assets of the experiment
\2 Actors: Administrators and Researchers
\2 Basic Flow:
\3 System will display experiment fields (name, description, qualifications, schedule time, schedule slots, and participant list)
\3 User will click on desired field to modify [Alt Flow 3]
\3 System will allow field that user chooses to be editable in line
\3 User will then change field as desired and click away from the field or save when finished
\3 System will update the database with the modified experiment information
\2 Alternate Flows:
\3 User cancels out of creating an experiment. System will return user to the page where user came from
\3 Saving an experiment fails
\3 User deletes an experiment. System will remove experiment from database after user confirmation and display a message to the user indicating this was successful
\2 Pre-conditions:
\3 User is an Administrator and/or a Researcher and has authenticated
\3 User chose experiment through one of the experiment views
\2 Post-conditions:
\3 System will have recorded the modifications to the experiment or the system will notify the user why the modification of the experiment failed
\2 Special Requirements:
\3 N/A

\end{outline}
\subsection{Reports}
\begin{outline}[enumerate]

\1 {\bf Name: List Experiment Participants}
\2 Brief description: Researcher logs in and views a list of all participants for a selected experiment.
\2 Actors: Researcher
\2 Basic flow
\3 Researcher logs in
\3 System displays table of researcher's experiments
\3 Researcher selects experiment from table
\3 System displays list of all participants for selected experiment
        \begin{table}
            \begin{tabular}{|l|l|l|l|}
                \hline
                Name       & Phone number   & Email address          & Date/time      \\ \hline
                First Last & (123) 123-4567 & first.last@example.com & 2011-9-1 13:00 \\
                First Last & (123) 123-4567 & first.last@example.com & 2011-9-1 13:00 \\
                First Last & (123) 123-4567 & first.last@example.com & 2011-9-1 13:00 \\
                First Last & (123) 123-4567 & first.last@example.com & 2011-9-1 13:00 \\
                First Last & (123) 123-4567 & first.last@example.com & 2011-9-1 13:00 \\
                \hline
            \end{tabular}
        \end{table}
\2 Alternate flows
\3 Researcher does not own any experiments
\4 (2) displays an empty table
\4 He cannot proceed past (2) until he creates an experiment or is added to another researcher's
\3 Selected experiment has no participants
\4 (4) displays an empty table
\4 Nothing is displayed in (4) until a participant signs up for the selected experiment
\2 Pre-conditions
\3 System is running
\3 System is in ready state
\3 Researcher has account with correct permissions/groups
\2 Post-conditions
\3 Researcher knows who is signed up to participate in his selected experiment or there are no experiments/participants
\2 Special Requirements:
\3 N/A


\1 {\bf Name: Cancel Experiment Appointment}
\2 Brief description: Participant logs in and cancels an appointment.
\2 Actors: Participant (User)
\2 Basic flow
\3 Participant logs in
\3 System displays table of participant's appointments
\3 Participant selects appointment from table
\3 System displays details for selected appointment
\3 Participant selects delete
\3 System displays confirmation prompt
\3 Participant selects confirm: appointment is deleted and system returns to (2) with an affirmation message
\3 Participant selects cancel: system returns to (5)
\2 Alternate flows
\3 Participant has no appointments
\4 (2) displays an empty table
\4 He cannot proceed past (2) until he signs up for an experiment
\2 Pre-conditions
\3 System is running
\3 System is in ready state
\3 Participant has account
\2 Post-conditions
\3 Participant deleted selected appointment or participant cancelled operation
\3 Researcher(s) owning said appointment's experiment are notified via email
\2 Special Requirements:
\3 N/A


\1 {\bf Name: Report Experiment Participant Lists}
\2 Brief Description:  When the user is a researcher, the user will be able to export a CSV file, filed with the Experiment name and participant and times.
\2 Actors: Researcher
\2 Basic Flow:
\3 The researcher will check what experiments to export to the CSV file from the list of experiments in the researcher side view
\3 The researcher will click “Export to CSV”
\3 The system will generate a CSV file from the selected experiment displaying the name of the experiment and the names of participants with their times
\3 The system will then start the download of the file to the researcher's computer
\3 When the system has completed 3 and 4, the system will display a message box “Export Complete!”
\3 The researcher will click “OK” or the exit button on the message box
\3 The system will return to the researcher side view.
\2 Alternative Flow:
\3 The researcher did not select any experiment.  An error window will appear.
\3 The system encounters an error when pulling data from the database. An error window will appear
\3 The system encounters any error when creating the CSV file. An error window will appear
\3 The system cannot download the file to the researcher’s computer. An error window will appear
\3 The researcher will deny the download of the CSV.  A message box will appear
\3 The user exits the browser
\2 Preconditions:
\3 The researcher must be logged in as a researcher
\3 The system is in the researcher side view
\3 The researcher must already have experiments scheduled
\2 Postconditions:
\3 The system is back in the researcher side view
\2 Feature mapping:
\3 Export Experiment Participant List

\1 {\bf Name: Calendar/List of All Experiments}
\2 Basic Description: The list will show the soonest to be conducted experiments (10) and will allow for a user to click and view more information on the experiment
\2 Basic Flow:
\3 The system loads the first 10 experiments
\3 The user can scroll down the list
\3 The user selects an experiment
\2 Alternative Flow:
\3The system errors when loading the first 10 experiments from the database
\3 The user goes to another website
\3 The user attempts to log into the researcher page and fails
\3 The user attempts to log into the researcher page and has success
\3 The user clicks on the next 10 experiments and repeats basic flow 1 to 3
\3 The user exits the browser
\2 Precondition:
\3 The user is on the web page
\2 Postcondidtion:
\3 The system is showing an experiment or the browser is on a new page
\2 Feature mapping:
\3 All calendar Experiments
\3 Browse Experiments
\3 Persistent Experiment Storage



\end{outline}
