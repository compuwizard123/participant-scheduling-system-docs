\section{Use Cases}
\subsection{Experiments}
\begin{outline}[enumerate]
\1 Name: Sign Up for Experiment
\2 Brief Description: Prospective participant chooses and signs up for an experiment.
\2 Actors: Participant (henceforth "user")
\2 Basic Flow: (User can return to a previous step with the "back" options on their browser.)  Each page has a login/logout button, which will follow the use case "Authentication" if clicked.
\3 Homepage: The system homepage has a button for login, and a table displaying currently running experiments.  Each experiment on the table can be clicked for further details.
\3 Step 1: User clicks on an experiment button.  This will navigate them to that experiment's page.
\4 Possible substep: The experiment listing table has options for sorting and filtering.  The user may click these buttons and enter filters without altering flow.
\3 Experiment Page: Each experiment has a webpage with its name, description, and a list of timeslots.  Each timeslot can be clicked to move to the confimation screen.  There is also a button to join the experiment which will send the user to the confirmation screen without a timeslot selected.
\3 Step 2: User clicks on a timeslot or the join button.  The system will navigate them to the confirmation page.
\4 Possible substep: The timeslot listing has options for sorting and filtering.  The user may click these buttons and enter filters without altering flow.
\3 Step 3: If the system attempts to move to the confirmation page but the user is not authenticated, it will follow the use case "Authentication" first.  Once they have successfully logged in, they will be sent directly to the confirmation page.
\3 Confirmation Page: This page can only be accessed while logged in.  It displays the experiment name, required qualifications, and a checkbox certifying that the user meets these requirements.  The timeslot list from the experiment page is also on this page, with the timeslot remaining selected if the user did so in step 2.  There is a large "Confirm Appointment" button at the bottom of the page.
\3 Step 4: The user checks the box, selects a timeslot if they have not already, and clicks the "Confirm Appointment" button.
\3 Final actions: The system will send the appointment data to the database and return the user to the homepage, with a popup confirming their successful registration.  The system will also send an email with the experiment and timeslot information to the account they used to register.
\2 Alternate Flows:
\3 User logs out while on the confirmation page.
\4 As a user cannot return to this page while logged out (the normal behavior for the Authentication use case), they will be shunted back to the experiment page.  Information provided in step 2 is not guaranteed to remain selected.
\3 Database reports an error when receiving appointment data.
\4 Unlikely to occur outside of concurrent modification (two people signing up at the same time), the system will shunt the user back to the confirmation page with the updated available timeslots and provide a popup explaining why this happened.


\1 Name: Add Experiment
\2 Brief Description: Experiments can be created by Administrators and Researchers
\2 Actors: Administrators and Researchers
\2 Basic Flow: (user can cancel at any time and follow Alt Flow 1)
\3 User must click on Add New Experiment link from the Administration "home" page
\3 System will display a screen with text boxes to enter experiment name, description,  and qualifications, multiple date/time choosers for the schedule times, and a drop down list to specify the length of the schedule slots
\3 User must enter the experiment information for name, description, qualifications, and schedule slots
\3 User can then begin setting up the schedule times by choosing date, begin, and end time for each slot they want to run the experiment
\3 User then must save the experiment by clicking the Save button
\3 System will then save the experiment to persistent storage and provide the user with confirmation that the experiment was created successfully and redirect user to all experiment view
\2 Alternate Flows:
\3 User cancels out of creating an experiment
\3 Saving an experiment fails
\2 Pre-conditions:
\3 User is an Administrator and/or a Researcher and has authenticated
\2 Post-conditions:
\3 System will have recorded the experiment or the system will notify the user why the creation of the experiment failed
\2 Special Requirements:
\3 N/A

\1 Name: Modify Experiment (user can cancel at any time and follow Alt Flow 1)
\2 Brief Description: Experiments can be modified by Administrators and Researchers to change all assets of the experiment
\2 Actors: Administrators and Researchers
\2 Basic Flow:
\3 System will display experiment fields (name, description, qualifications, schedule time, schedule slots, and participant list)
\3 User will click on desired field to modify [Alt Flow 3]
\3 System will allow field that user choose to be editable in line
\3 User will then change field as desired and click out or save when finished
\3 System will update the database with the modified experiment information
\2 Alternate Flows:
\3 User cancels out of creating an experiment. System will return user to the page where user came from
\3 Saving an experiment fails
\3 User deletes an experiment. System will remove experiment from database after user confirmation and display a message to the user indicating this was successful
\2 Pre-conditions:
\3 User is an Administrator and/or a Researcher and has authenticated
\3 User choose experiment through one of the experiment views
\2 Post-conditions:
\3 System will have recorded the modifications to the experiment or the system will notify the user why the modification of the experiment failed
\2 Special Requirements:
\3 N/A
\end{outline}