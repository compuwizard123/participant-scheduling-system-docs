\documentclass{article}
\usepackage{fullpage}
\title{Milestone 1}
\author{Trey Cahill, Chris Gropp, Samad Jawaid, Kevin Risden}
\begin{document}
\maketitle
\section{Executive Summary}
\section{Introduction}
The Human-Computer Interaction Lab at the University of Wisconsin-Madison wants a web-based system to better manage the scheduling of participants for their studies.  These studies range from one-on-one experiments to group interactions, and many of them involve the robot used by the lab.  Currently, each researcher arranges studies independently via email and is responsible for scheduling rooms, avoiding conflicts, and notifying participants of changes; unifying this information onto one system simplifies all of these tasks.  To the client, the most important benefit of a unified system is the ability for participants to easily browse all available experiments, which is not possible over email.  However, a variety of other functionality should be integrated into this utility to take advantage of the unity of information; most notable is recognizing room conflicts when scheduling studies, since the lab has only one robot and it cannot be moved.
\section{Client Background}
What the client does? It must describe what the client does, both in a broad overall sense and in day-to-day affairs, and describe how the project will fit into the day-to-day workings for the client. Use this section to provide a general overview; specifics can be given in the next section.
\section{Current System}
A brief description of the current system (if any) and identify its features
\section{User/Stakeholder Description}
Use this section to identify your users (other stakeholders) and how they will use the system. Must contain the following subsections
\subsection{User/Stakeholder Profiles}
See page 439 of requirements text
\subsection{User Environment}
See page 439 of requirements text
\subsection{Key Needs}
List the key problems or needs as perceived by the user. For each user need, answer the following questions
\subsubsection{What is the need?}
Use the problem statement template shown in Page 46 of requirements text
\subsubsection{How is it solved now?}
\subsubsection{What is a possible solution?}
\subsection{Alternatives and Competition}
See page 440 of requirements text
\section{Product Overview}
Use it to provide a high level overview of your proposed system. It must contain the following subsections
\subsection{Product perspective}
See page 440 of requirements text
\subsection{Elevator Statement}
pitch as discussed in class
\subsection{Summary of Capabilities}
See page 441 of requirements text
\subsection{Assumptions and Dependencies}
See page 441 of requirements text
\subsection{Rough Estimate of the Cost}
\section{Features}
\begin{table}[!hb]\footnotesize
    \begin{tabular}{|p{2.5cm}|c|c|p{1.25cm}|p{1.5cm}|p{1.5cm}|p{1.25cm}|p{3.25cm}|}
        \hline
        Feature & Status & Priority & Effort & Risk  & Stability & Target Release & Reason \\
        \hline
        Browse Experiment & Approved & Critical & Medium & Medium - High & Low - Medium & 1st release & Lets experiments be advertised better and to display the experiments \\
        \hline
        Store Experiments & Approved & Critical & Medium & Low   & High  & 1st release & Store experiment for the data to be web based. \\
        \hline
        Levels of Authentication & Approved & Critical & Medium & High  & Medium & 1st release  & Have levels of admins, workers and participants in order to control privacy issues and other sensitive data \\
        \hline
        Schedule Experiment & Approved & Critical & Medium & Medium - High & Low - Medium & 1st release   & Need to schedule experiments in order to browse them \\
        \hline
        Filter Experiments & Approved & Useful & Low-Medium & Low   & High  & 2nd release & Filter the experiments when browsing according to Time, Date, Payment, etc. \\
        \hline
        View All Participants for an Individual Experiment & Approved & Important & Low-Medium & low   & High  & 2nd release  & View all of the participants by admins and workers only of individual experiments \\
        \hline
        Admin Back End & Approved & Useful & Medium - High & Low   & Low   & 2nd or 3rd release  & A back end for the admins to do their duties from \\
        \hline
        Edit/Modify Schedule Slot & Approved & Useful & Medium & Medium & Medium & 2nd or 3rd release   & Modify or Edit schedule from a participants view \\
        \hline
        Edit/Modify Schedule Slot & Approved & Important & Medium & Medium & Medium & 2nd release  & Modify or Edit schedule from a workers/admins view \\
        \hline
        Notify Participant Reminder & Approved & Useful & Medium & Low   & High  & 4th release  & Send an email or text reminding participants for their experiments \\
        \hline
        Form to get User Info & Approved & Useful & Low   & Low   & High  & 4th release  & A form to gather participant Info \\
        \hline
        Admin Report & Approved & Useful & High  & Low   & High  & 4th release  & Reports on experiments scheduled with an option for Individual experiments reports \\
        \hline
        Ease of Participant Scheduling & Approved & Useful & Medium & Low   & Low   & 4th release  & Make scheduling a near 1 click process \\
        \hline
        Overall Schedule & Approved & Useful & Medium & Low   & Low   & 4th release  & Have an overall schedule viewer \\
        \hline
        Cancel Schedule Slot & Approved & Important & Low   & Low   & High  & 4th release  & Allow for participants to cancel schedule slots for appointments \\
        \hline
        Remove Experiments & Approved & Important & Low   & Low   & High  & 4th release  & Allow for workers or admins to remove schedules \\
        \hline
        Tracking of Consent Payment Forms & Proposed & Useful & Medium & Low   & Medium & 5th release  & Allow for workers to check off participants when filling out consent/payment forms \\
        \hline
        User Report & Proposed & Useful & Medium & Low   & Medium & 5th release & Allow participants to have a report on new experiments \\
        \hline
        Accounts & Proposed & Useful & High  & Low   & Medium & TBD & Accounts for users \\
        \hline
        Prevent Scheduling Conflicts & Proposed & Useful & High  & Low   & Medium & TBD & Prevent participants from scheduling 2 experiments at once \\
        \hline
        Prevent Scheduling Conflicts & Proposed & Useful & High  & Low   & Medium & TBD  & Prevent 2 rooms from being scheduled at the same time \\
        \hline
        Install Scripts & Proposed & Useful & High  & Low   & Low   & TBD & Install scripts for installation \\
        \hline
        Documentation for Maintenance and User & Proposed & Useful & High  & High  & Low   & TBD  & Documentation \\
        \hline
    \end{tabular}
\end{table}
\section{Solution Constraints}
\begin{table}[!hb]\footnotesize
    \begin{tabular}{|r|p{5.5cm}|p{5.5cm}|}
        \hline
        Source & Constraint & Rationale \\
        \hline
        Systems Mandate & Must be able to be ran on Red Hat Enterprise Linux Server 6.1. & This is the operating system that the client currently uses. \\
        \hline
        Technology Mandate & Must use PHP or Python as the programming languages & These are the languages supported by the client \\
        \hline
        Databases Mandate & Must use MySQL or PostgreSQL database management systems & These are the database management systems supported by the client. \\
        \hline
        Time  & The time constraint on the project is the end of Second Term (Rose Hulman time) & At this point the group is reduced to 1 person. \\
        \hline
        Equipment Budget & No new equipment can be bought for the project & The software will be place on an existing server and we have no budget. \\
        \hline
        Privacy & Participants must not be able to see who else is participating in a project. & Privacy is key for the experiments. \\
        \hline
    \end{tabular}
\end{table}
\end{document}
