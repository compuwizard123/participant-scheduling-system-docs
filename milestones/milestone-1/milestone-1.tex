\documentclass{article}
\usepackage{geometry} % to change the page dimensions
\geometry{letterpaper} % or letterpaper (US) or a5paper or....
\geometry{margin=1in} % for example, change the margins to 2 inches all round

\title{Milestone 1}
\author{Trey Cahill, Chris Gropp, Samad Jawaid, Kevin Risden}
\begin{document}
\maketitle
\section{Executive Summary}
\section{Introduction}
The Human-Computer Interaction Lab at the University of Wisconsin-Madison wants a web-based system to better manage the scheduling of participants for their studies.  These studies range from one-on-one experiments to group interactions, and many of them involve the robot used by the lab.  Currently, each researcher arranges studies independently via email and is responsible for scheduling rooms, avoiding conflicts, and notifying participants of changes; unifying this information onto one system simplifies all of these tasks.  To the client, the most important benefit of a unified system is the ability for participants to easily browse all available experiments, which is not possible over email.  However, a variety of other functionality should be integrated into this utility to take advantage of the unity of information; most notable is recognizing room conflicts when scheduling studies, since the lab has only one robot and it cannot be moved.

\section{Client Background}
What the client does? It must describe what the client does, both in a broad overall sense and in day-to-day affairs, and describe how the project will fit into the day-to-day workings for the client. Use this section to provide a general overview; specifics can be given in the next section.

\section{Current System}
A brief description of the current system (if any) and identify its features

\section{User/Stakeholder Description}
\subsection{User/Stakeholder Profiles}
The stakeholders for our system fall into two distinct categories, participants and the people in the research lab. These categories then fall into a few subsections that will be described in detail. The participants themselves fall into the categories of technical and nontechnical users. For the technical users they may some a high degree of technical background and will be wanting a rich user experience. Furthermore these users will want the system to be more robust and provide a large number of features to make them feel comfortable. Success to them would include a system that they can possibly custom and work into their already technical work flow. The nontechnical users may or may have had experience with any sort of web interface. This provides an extra challenge for our solution because we must make the interface intuitive enough that people who are not comfortable with computers can use. A success for the nontechnical user would be the ability for them to use the software in a way that does not interfere with their normal daily work flow and they can overcome any hesitancy they have with using computers to schedule their participation in an experiment.
The other class of stakeholders involves people in the research lab. Since the lab consists of undergraduate students, graduate students, and professors, each of their use of the system varies. Furthermore, since the research lab will be controlling and maintaining the system once in place their is a subclass of the people in the research lab which act as both administrators and maintenance technicians. Since the research lab is involved with Human-Computer Interaction, many of the people in the lab will be experienced with computers. However, from the client we learned that a few outside researchers who may use the system we develop may not have a large degree of technical knowledge. Therefore we must design the system in a way that permits researchers, students, and professors to setup experiments and look at participant lists in an effective manner for all of them. Since these people have been using a rudimentary system before, we must make the new system capable of integrating into their work flow without causing them any more work than before. The success for the researchers, students, and professors comes from being able to schedule participants in a way that works effectively and simply from their point of view. The administrators and maintenance people from the research lab we know are all technically savvy users. This means the system must be simple to maintain and possible expand at a later date if needed. The key responsibilities for the administrator include being able to add and remove other users from the system and ensure that the system is functioning correctly. The maintenance people must ensure that the system is stable and in a working condition when required. The success for the administrator can be defined as having an intuitive management interface that provides them the flexibility they need to perform their duties. For the maintenance people, success can be defined as having a system that is stable and requires little work to keep it operational even over a large number of years.

\subsection{User Environment}
The user environment is basically the same between the classes of users listed above. The working environment consists of mainly just one user at a time on their own machine joining an experiment. This will stay constant but there may be many users at once on separate computers for any number of experiments. The task length should be fairly minimal and involves them finding an experiment they would like to participant in and then signing up. The entire length of the activity should only take a few minutes as to not inconvenience the user. The only environmental constraint that will be imposed on the user is that they must use a web browser of some sort. This means that the user will be required to have some device that has internet access. The system today involves the user going to a website to find the contact email and then sending the contact an email specifying that they would like to participant in an experiment. The users may have some sort of calendar system that they use so we may have to integrate with it but it is not a hard requirement but instead a possible feature.

\subsection{Key Needs}
List the key problems or needs as perceived by the user. For each user need, answer the following questions

\subsubsection{What is the need?}
Use the problem statement template shown in Page 46 of requirements text
\begin{table}[!hb]
    \begin{tabular}{|l|l|}
        \hline
        The problem of... &  \\ \hline
        Affects... &  \\ \hline
        And results in... &  \\ \hline
        Benefits of a solution... &  \\ \hline
    \end{tabular}
\end{table}
\subsubsection{How is it solved now?}
\subsubsection{What is a possible solution?}

\subsection{Alternatives and Competition}
\subsubsection{Google Calendar Appointment Slots}
The issues the client mentioned about Google Calendar appointment slots was that their was no way to list all experiments in an easy format and no way to do multiple experiments at once. Furthermore, it would be a hassle to have to compile all the appointment slots into a list of participants for the experiment. The process to use Google Calendar appointment slots was also very manual as for each experiment the researcher would need to go through and outline all available times. The approach also did not work well for participants since they would need to navigate to each calendar separately to look at all available times for an experiment.

\section{Product Overview}
This section provides a high-level view of the product capabilities, interfaces to other applications, and system configurations.

\subsection{Product perspective}
The participant scheduling system will be a new product. It will be used to schedule experiments and participants in the Human-Computer Interaction Lab at the University of Wisconsin-Madison. The product is independent and totally self-contained; it is not the component of a larger system.

\subsection{Elevator Statement}
For the researchers in the Human-Computer Interaction Lab at the University of Wisconsin-Madison who currently schedule experiments and participants with rudimentary tools such as pencil and paper, email, or Google Calendar, the participant scheduling system will be a web application that will streamline the lab's scheduling process. Unlike current solutions, this application will be the same for every researcher, so it will also be easier for participants to be a part of multiple experiments.

\subsection{Summary of Capabilities}
Here are the major benefits and features the product will provide.
\begin{table}[!hb]
    \begin{tabular}{|l|l|}
        \hline
        Customer Benefit & Supporting Feature \\ \hline
        List of participants for an experiment & Reports \\ \hline
        Room availability (avoid conflicts) & Overall lab schedule \\ \hline
        Simple sign up & Intuitive user interface \\ \hline
        Track all experiments & Experiments manager \\ \hline
        Access from anywhere at anytime & Web application \\ \hline
    \end{tabular}
\end{table}
\subsection{Assumptions and Dependencies}
\begin{itemize}
\item The participant scheduling system will be a web application.
\item The server has the necessary operating system and software.
\item There is no integration with any other system.
\item There is no import of existing data.
\end{itemize}
\subsection{Rough Estimate of the Cost}
There is no monetary cost for this project, because the software development, as part of a college class, is free. Furthermore, the client will be provided with free servers through the University of Wisconsin-Madison for the finished product. The client will perform maintainence and management on their own.
\section{Features}
\begin{table}[!hb]\footnotesize
    \begin{tabular}{|p{2.5cm}|c|c|p{1.25cm}|p{1.5cm}|p{1.5cm}|p{1.25cm}|p{3.25cm}|}
        \hline
        Feature & Status & Priority & Effort & Risk  & Stability & Target Release & Reason \\
        \hline
        Browse Experiment & Approved & Critical & Medium & Medium - High & Low - Medium & 1st release & Lets experiments be advertised better and to display the experiments \\
        \hline
        Store Experiments & Approved & Critical & Medium & Low   & High  & 1st release & Store experiment for the data to be web based. \\
        \hline
        Levels of Authentication & Approved & Critical & Medium & High  & Medium & 1st release  & Have levels of admins, workers and participants in order to control privacy issues and other sensitive data \\
        \hline
        Schedule Experiment & Approved & Critical & Medium & Medium - High & Low - Medium & 1st release   & Need to schedule experiments in order to browse them \\
        \hline
        Filter Experiments & Approved & Useful & Low-Medium & Low   & High  & 2nd release & Filter the experiments when browsing according to Time, Date, Payment, etc. \\
        \hline
        View All Participants for an Individual Experiment & Approved & Important & Low-Medium & low   & High  & 2nd release  & View all of the participants by admins and workers only of individual experiments \\
        \hline
        Admin Back End & Approved & Useful & Medium - High & Low   & Low   & 2nd or 3rd release  & A back end for the admins to do their duties from \\
        \hline
        Edit/Modify Schedule Slot & Approved & Useful & Medium & Medium & Medium & 2nd or 3rd release   & Modify or Edit schedule from a participants view \\
        \hline
        Edit/Modify Schedule Slot & Approved & Important & Medium & Medium & Medium & 2nd release  & Modify or Edit schedule from a workers/admins view \\
        \hline
        Notify Participant Reminder & Approved & Useful & Medium & Low   & High  & 4th release  & Send an email or text reminding participants for their experiments \\
        \hline
        Form to get User Info & Approved & Useful & Low   & Low   & High  & 4th release  & A form to gather participant Info \\
        \hline
        Admin Report & Approved & Useful & High  & Low   & High  & 4th release  & Reports on experiments scheduled with an option for Individual experiments reports \\
        \hline
        Ease of Participant Scheduling & Approved & Useful & Medium & Low   & Low   & 4th release  & Make scheduling a near 1 click process \\
        \hline
        Overall Schedule & Approved & Useful & Medium & Low   & Low   & 4th release  & Have an overall schedule viewer \\
        \hline
        Cancel Schedule Slot & Approved & Important & Low   & Low   & High  & 4th release  & Allow for participants to cancel schedule slots for appointments \\
        \hline
        Remove Experiments & Approved & Important & Low   & Low   & High  & 4th release  & Allow for workers or admins to remove schedules \\
        \hline
        Tracking of Consent Payment Forms & Proposed & Useful & Medium & Low   & Medium & 5th release  & Allow for workers to check off participants when filling out consent/payment forms \\
        \hline
        User Report & Proposed & Useful & Medium & Low   & Medium & 5th release & Allow participants to have a report on new experiments \\
        \hline
        Accounts & Proposed & Useful & High  & Low   & Medium & TBD & Accounts for users \\
        \hline
        Prevent Scheduling Conflicts & Proposed & Useful & High  & Low   & Medium & TBD & Prevent participants from scheduling 2 experiments at once \\
        \hline
        Prevent Scheduling Conflicts & Proposed & Useful & High  & Low   & Medium & TBD  & Prevent 2 rooms from being scheduled at the same time \\
        \hline
        Install Scripts & Proposed & Useful & High  & Low   & Low   & TBD & Install scripts for installation \\
        \hline
        Documentation for Maintenance and User & Proposed & Useful & High  & High  & Low   & TBD  & Documentation \\
        \hline
    \end{tabular}
\end{table}
\section{Solution Constraints}
\begin{table}[!hb]\footnotesize
    \begin{tabular}{|r|p{5.5cm}|p{5.5cm}|}
        \hline
        Source & Constraint & Rationale \\
        \hline
        Systems Mandate & Must be able to be ran on Red Hat Enterprise Linux Server 6.1. & This is the operating system that the client currently uses. \\
        \hline
        Technology Mandate & Must use PHP or Python as the programming languages & These are the languages supported by the client \\
        \hline
        Databases Mandate & Must use MySQL or PostgreSQL database management systems & These are the database management systems supported by the client. \\
        \hline
        Time  & The time constraint on the project is the end of Second Term (Rose Hulman time) & At this point the group is reduced to 1 person. \\
        \hline
        Equipment Budget & No new equipment can be bought for the project & The software will be place on an existing server and we have no budget. \\
        \hline
        Privacy & Participants must not be able to see who else is participating in a project. & Privacy is key for the experiments. \\
        \hline
    \end{tabular}
\end{table}
\end{document}
