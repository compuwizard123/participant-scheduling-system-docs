\documentclass[]{article}

\usepackage[parfill]{parskip} % to begin paragraphs with an empty line rather than an indent
\usepackage{geometry} % to change the page dimensions
\usepackage{graphicx}
\geometry{letterpaper} % or letterpaper (US) or a5paper or....
\geometry{margin=1in} % for example, change the margins to 2 inches all round

\usepackage{outlines}

\title{Use Cases}
\author{Kevin Risden}

\begin{document}
\maketitle
\section{Use Cases}
\subsection{Experiments}
\begin{outline}[enumerate]
\1 Name: Add Experiment
\2 Brief Description: Experiments can be created by Administrators and Researchers
\2 Actors: Administrators and Researchers
\2 Basic Flow: (user can cancel at any time and follow Alt Flow 1)
\3 User must click on Add New Experiment link from the Administration "home" page
\3 System will display a screen with text boxes to enter experiment name, description,  and qualifications, multiple date/time choosers for the schedule times, and a drop down list to specify the length of the schedule slots
\3 User must enter the experiment information for name, description, qualifications, and schedule slots
\3 User can then begin setting up the schedule times by choosing date, begin, and end time for each slot they want to run the experiment
\3 User then must save the experiment by clicking the Save button
\3 System will then save the experiment to persistent storage and provide the user with confirmation that the experiment was created successfully and redirect user to all experiment view
\2 Alternate Flows:
\3 User cancels out of creating an experiment
\3 Saving an experiment fails
\2 Pre-conditions:
\3 User is an Administrator and/or a Researcher and has authenticated
\2 Post-conditions:
\3 System will have recorded the experiment or the system will notify the user why the creation of the experiment failed
\2 Special Requirements:
\3 N/A

\1 Name: Modify Experiment (user can cancel at any time and follow Alt Flow 1)
\2 Brief Description: Experiments can be modified by Administrators and Researchers to change all assets of the experiment
\2 Actors: Administrators and Researchers
\2 Basic Flow:
\3 System will display experiment fields (name, description, qualifications, schedule time, schedule slots, and participant list)
\3 User will click on desired field to modify [Alt Flow 3]
\3 System will allow field that user choose to be editable in line
\3 User will then change field as desired and click out or save when finished
\3 System will update the database with the modified experiment information
\2 Alternate Flows:
\3 User cancels out of creating an experiment. System will return user to the page where user came from
\3 Saving an experiment fails
\3 User deletes an experiment. System will remove experiment from database after user confirmation and display a message to the user indicating this was successful
\2 Pre-conditions:
\3 User is an Administrator and/or a Researcher and has authenticated
\3 User choose experiment through one of the experiment views
\2 Post-conditions:
\3 System will have recorded the modifications to the experiment or the system will notify the user why the modification of the experiment failed
\2 Special Requirements:
\3 N/A
\end{outline}
\end{document}
