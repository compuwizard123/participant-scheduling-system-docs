\documentclass{article}
\title{Use Cases}
\author{Samad Jawaid}

\begin{document}
\maketitle

\section{Feature 1}
\begin{enumerate}
\item[A.] Name: List Experiment Participants
\item[B.] Brief description: Researcher logs in and views a list of all participants for a selected experiment.
\item[C.] Actors: Researcher
\item[D.] Basic flow
    \begin{enumerate}
    \item[1.] Researcher logs in
    \item[2.] System displays table of researcher's experiments
    \item[3.] Researcher selects experiment from table
    \item[4.] System displays list of all participants for selected experiment
        \begin{table}
            \begin{tabular}{|l|l|l|l|}
                \hline
                Name       & Phone number   & Email address          & Date/time      \\ \hline
                First Last & (123) 123-4567 & first.last@example.com & 2011-9-1 13:00 \\
                First Last & (123) 123-4567 & first.last@example.com & 2011-9-1 13:00 \\
                First Last & (123) 123-4567 & first.last@example.com & 2011-9-1 13:00 \\
                First Last & (123) 123-4567 & first.last@example.com & 2011-9-1 13:00 \\
                First Last & (123) 123-4567 & first.last@example.com & 2011-9-1 13:00 \\
                \hline
            \end{tabular}
        \end{table}
    \end{enumerate}
\item[E.] Alternate flows
    \begin{itemize}
    \item Researcher does not own any experiments
        \begin{itemize}
        \item (2) displays an empty table
        \item He cannot proceed past (2) until he creates an experiment or is added to another researcher's
        \end{itemize}
    \item Selected experiment has no participants
        \begin{itemize}
        \item (4) displays an empty table
        \item Nothing is displayed in (4) until a participant signs up for the selected experiment
        \end{itemize}
    \end{itemize}
\item[F.] Pre-conditions
    \begin{itemize}
    \item System is running
    \item System is in ready state
    \item Researcher has account with correct permissions/groups
    \end{itemize}
\item[G.] Post-conditions
    \begin{itemize}
    \item Researcher knows who is signed up to participate in his selected experiment or there are no experiments/participants
    \end{itemize}
\end{enumerate}

\section{Feature 2}
\begin{enumerate}
\item[A.] Name: Cancel Experiment Appointment
\item[B.] Brief description: Participant logs in and cancels an appointment.
\item[C.] Actors: Participant (User)
\item[D.] Basic flow
    \begin{enumerate}
    \item[1.] Participant logs in
    \item[2.] System displays table of participant's appointments
    \item[3.] Participant selects appointment from table
    \item[4.] System displays details for selected appointment
    \item[5.] Participant selects delete
    \item[6.] System displays confirmation prompt
    \item[7.] Participant selects confirm: appointment is deleted and system returns to (2) with an affirmation message
    \item[8.] Participant selects cancel: system returns to (5)
    \end{enumerate}
\item[E.] Alternate flows
    \begin{itemize}
    \item Participant has no appointments
        \begin{itemize}
        \item (2) displays an empty table
        \item He cannot proceed past (2) until he signs up for an experiment
        \end{itemize}
    \end{itemize}
\item[F.] Pre-conditions
    \begin{itemize}
    \item System is running
    \item System is in ready state
    \item Participant has account
    \end{itemize}
\item[G.] Post-conditions
    \begin{itemize}
    \item Participant deleted selected appointment or participant cancelled operation
    \item Researcher(s) owning said appointment's experiment are notified via email
    \end{itemize}
\end{enumerate}

\end{document}
